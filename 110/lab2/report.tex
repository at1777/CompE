% A skeleton file for producing Computer Engineering reports
% https://kgcoe-git.rit.edu/jgm6496/KGCOEReport_template

\documentclass[CMPE]{KGCOEReport}

% The following should be changed to represent your personal information
\newcommand{\classCode}{CMPE 110}  % 4 char code with number
\newcommand{\name}{Andrei Tumbar}
\newcommand{\LabSectionNum}{05}
\newcommand{\LabInstructor}{Mark Byers}	% The slash is to tell LaTeX that the period is between words
												% not sentences so it spaces correctly. It won't appear in the
												% final PDF
\newcommand{\TAs}{Rickie Yeung \\ Tommy Li}
\newcommand{\LectureSectionNum}{01}
\newcommand{\LectureInstructor}{Dr.\ Soldavini}
\newcommand{\exerciseNumber}{2}
\newcommand{\exerciseDescription}{Ohm's Law}
\newcommand{\dateDone}{September 05, 2019}
\newcommand{\dateSubmitted}{September 19, 2019}

\usepackage{circuitikz}
\usepackage{multirow}
\usepackage{float}
\usepackage{array}
\usepackage{pgfplots, pgfplotstable}
\usepackage{amsmath}
\usepackage{amssymb}

\begin{document}
\maketitle

\section*{Abstract}

Understanding Ohm's Law as well as Kirchoff's loop laws is important and vital when designing circuits. This laboratory exercise exposes the relationship between voltage, current, and resistance in both a parallel and series configuration. A power supply supplied a variable voltage across 3 resistors, two in parallel and the third in series. An effective resistance was calculated and compared to a measured resistance. Potential drop across resistors was measured at every integer voltage from 2-5V. A Voltage vs. Current graph was generated illustrating the linear relationship between the two variables for ohmic resistors. This exercise proves Kirchoff's loop laws and Ohm's law.

\section*{Design Methodology}

The three smallest resistors were determined. Resistors were named $R_1$, $R_2$, and $R_3$ in order by increasing resistance. The actual resistance was measured using the multimeter's ohmeter function (See table 1).


\begin{figure}[h!]
\begin{center}
\begin{circuitikz}
\draw
(0,0) to[battery1, l=V] (0,4) --
(0,4) to[R, l_=$R_1$, i_=$I_1$, v^=$V_1$] (4,4) --
(6,4) to[R, l_=$R_3$, i_=$I_3$, v^={$V_2$ $V_3$}] (6,0) -- (4,0)
(4,4) to[R, l_=$R_2$, i_=$I_2$] (4,0) --
(0,0)
;
\end{circuitikz}
\end{center}
\caption{Circuit configuration given $R_1$, $R_2$, and $R_3$ have acceding resistances.}
\end{figure}

A circuit was wired on a breadboard such that $R_2$ and $R_3$ were placed in parallel to each other and in series to $R_1$. Figure 1 shows an example circuit in the correct configuration. With the power supply voltage set to 5V, potential drop across $R_1$ and $R_2$ was taken. This step was repeated for 2V, 3V, and 4V across the power supply.

Current through all three resistors was measured at a 5V total potential. An effective resistance for the entire circuit was calculated. First the two parallel branches were combined through equation \ref{eq:1}:

\begin{gather}
\label{eq:1}
R_{eq} = \frac{1}{ \sum\frac{1}{R_i} }
\end{gather}

Resistance of $R_1$ was added to the value about yielding a total effective resistance of the circuit.  

Finally, current through $R_3$ was measured from 2-5V at 1V intervals from the power supply and recorded. Using this data, a graph was generated for Voltage vs. Current through $R_3$.

\section*{Results and Analysis}
\subsection*{Resistance}

Table 1 contains measured resistance for each resistor. Rated resistance and tolerance are also shown.

\begin{table}[h]
\caption{Measurements taken for all three resistors.}
\begin{center}
\begin{tabular}{cc|c|c}
& \multicolumn{2}{c}{Banded Values} & Measured Values  \\\cline{2-4}
\multirow{2}{*}{}
& Rated Resistance & Rated Tolerance & Actual Resistance \\
& ($\Omega$) & (\%) & ($\Omega$) \\\cline{2-4}
$R_1$ & 330 & $\pm$ 5 & 328.00 \\\cline{2-4}
$R_2$ & 3330 & $\pm$ 5 & 3396.00 \\\cline{2-4}
$R_3$ & 5600 & $\pm$ 5 & 5598.80 \\\cline{2-4}
\end{tabular}
\end{center}
\end{table}

Measured resistance in $R_1$ $R_2$ and $R_3$ showed great accuracy to that of the rated resistance. 

\subsection*{Voltage}
Table 2 shows the data taken for the voltage drop across $R_1$ and $R_2$ for varying power supply voltage from 2V to 5V.

{
\renewcommand{\arraystretch}{1.4}
\begin{table}[h]
\caption{Voltage measured across $R_1$ and $R_2$.}
\begin{center}
\begin{tabular}{c>{\centering\arraybackslash}p{2cm}|>{\centering\arraybackslash}p{2cm}|>{\centering\arraybackslash}p{2cm}}
& $V_{PS}$ & $V_{R_1}$ & $V_{R_2}$ \\\cline{2-4}
\textasciitilde2V & 1.997 & 0.268 & 1.728 \\\cline{2-4}
\textasciitilde3V & 2.997 & 0.402 & 2.592 \\\cline{2-4}
\textasciitilde4V & 3.998 & 0.536 & 3.457 \\\cline{2-4}
\textasciitilde5V & 4.999 & 0.670 & 4.320
\end{tabular}
\end{center}
\end{table}
}

As expected, the voltage across $R_1$ and $R_2$ scale proportionally to the voltage across the power supply. 

\subsection*{Current}
{
\renewcommand{\arraystretch}{1.4}
\begin{table}[h]
\caption{Voltage measured across $R_1$ and $R_2$.}
\begin{center}
\begin{tabular}{>{\centering\arraybackslash}p{2cm}|>{\centering\arraybackslash}p{2cm}|>{\centering\arraybackslash}p{2cm}}
$I_1$ (mA) & $I_2$ (mA) & $I_3$ (mA) \\\hline
2.00 & 1.27 & 0.77
\end{tabular}
\end{center}
\end{table}
}

Current through any of the three resistors yielded very low numbers because the effective resistance was so high. Effective resistance came out to be $2441.84 \Omega$ calculated from:

\begin{gather}
R_{eq} = R_1 + \frac{1}{ \frac{1}{R_2} + \frac{1}{R_3} } = 2441.84\Omega
\end{gather}

\subsection*{Results}

Table 4 was generated by measuring the current through $R_3$ and the power supply at every 1V interval from 2-5V. The last column was generated by applying Ohm's Law $V=IR$.

\begin{table}[htb]
\caption{Table to test captions and labels}
\begin{center}
\begin{tabular}{c|>{\centering\arraybackslash}p{2cm}|>{\centering\arraybackslash}p{2cm}|>{\centering\arraybackslash}p{2cm}|>{\centering\arraybackslash}p{2cm}|>{\centering\arraybackslash}p{2cm}}
Voltage & V\textsubscript{PS} & V\textsubscript{R\textsubscript{3}} & I\textsubscript{PS} & I\textsubscript{R\textsubscript{3}} & R\textsubscript{3} \\
(V) & (V) & (V) & (mA) & (mA) & ($\Omega$) \\
\hline
2 & 1.997 & 1.728 & 0.815 & 0.308 & 5616.88  \\
3 & 2.997 & 2.592 & 1.220 & 0.463 & 5593.95 \\
4 & 3.998 & 3.475 & 1.630 & 0.617 & 5607.78 \\
5 & 4.999 & 4.320 & 2.040 & 0.771 & 5603.11
\end{tabular}
\end{center}
\end{table}

\pgfplotstableread{
X Y
0.308 1.728
0.462 2.592
0.617 3.475
0.771 4.320
}\datatable

\begin{figure}[H]
\begin{center}
\begin{tikzpicture}
\begin{axis}[
    xlabel={Current [mA]},
    ylabel={Voltage [V]},
    xmin=0.2, xmax=0.8,
    ymin=1, ymax=5,
    xtick={.2,.4,.6,.8},
    ytick={2,3,4,5},
    legend pos=north west,
    ymajorgrids=true,
    grid style=dashed,
]
 
\addplot[only marks, mark = *] table {\datatable};
\addplot [thick] table[
    y={create col/linear regression={y=Y}}
] {\datatable};
\addlegendentry{$data$}
\addlegendentry{%
$\pgfmathprintnumber[precision=3]{\pgfplotstableregressiona} \cdot x
\pgfmathprintnumber[print sign]{\pgfplotstableregressionb}$}

\end{axis}
\end{tikzpicture}
\caption{Voltage over $R_3$ vs Current through $R_3$}
\end{center}
\end{figure}

Figure 2 displays the data from table 4 for $I_{R_3}$ on the X-axis and $V_{R_3}$ on the Y-axis. The slope in figure 2 represents the resistance of $R_3$ in $k\Omega$ because according to Ohm's Law $R = \frac{V}{I}$.

\section*{Conclusion}

This laboratory exercise was to expose the relationship between voltage, current, and resistance in both a parallel and series circuit using Kirchoff’s loop laws and Ohm’s Law. Using the power supply, varying voltages were applied across three resistors, two in parallel and the third in series. Using the multimeter, the resistance, current, and voltage was taken across the power supply and resistors for 2, 3, 4, and 5 volts. $R_1$, $R_2$, and $R_3$ showed great accuracy to their banded resistances compared to their actual measured resistances from the multimeter. The voltage across $R_1$ and $R_2$ were proportional to the voltage across the power supply, proving Kirchoff’s voltage loop law. Because the linear regression closely correlated to the data, Ohm’s Law is proven. This exercise was effective in proving both Kirchoff’s loop laws and Ohm’s Law.

\end{document}
